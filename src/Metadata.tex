% !TEX root = ../Main.tex
% ==============================================================================
% Document-Specific General Settings
% ==============================================================================
% Description: This file contains all document-specific settings including
%              title, author information, abstract, keywords, and other
%              metadata required for the title page and document properties.
% Author: Frederik Beimgraben
% Date: 23.11.2025
% ==============================================================================

% ------------------------------------------------------------------------------
% Watermark and Date Configuration
% ------------------------------------------------------------------------------
% Configure document watermark and creation date
% These settings affect the document's visual presentation and metadata
% ------------------------------------------------------------------------------

% Watermark text (leave empty for no watermark)
% Example: \newcommand{\waterMarkText}{DRAFT}
% Example: \newcommand{\waterMarkText}{CONFIDENTIAL}
\newcommand{\waterMarkText}{}
\newcommand{\lastChanged}{\hl{XX. Monat 20XX}}
\newcommand{\acceptedUniversity}{\hl{XX. Monat 20XX}}

% Document creation date
% Format: DD.MM.YYYY
% This date appears on the title page and in document metadata
\createdon{\lastChanged}

% ------------------------------------------------------------------------------
% Document Title
% ------------------------------------------------------------------------------
% Main title of the document
% This appears on the title page and in PDF metadata
% For multi-line titles, use \\ for line breaks
% ------------------------------------------------------------------------------
\title{Organisationssatzung der\newline Studierendenschaft der Hochschule Reutlingen}

% ------------------------------------------------------------------------------
% Title Page Data Fields
% ------------------------------------------------------------------------------
% Configure all fields that appear on the title page
% Use \AddTitlePageDataLine{Label}{Content} for each field
% Use \AddTitlePageDataSpace{dimension} to add vertical spacing
% Use \newline within content for line breaks
% ------------------------------------------------------------------------------

% --- Author Information ---
% Student name, semester, and contact information
\author{STUPA der Hochschule Reutlingen}
\AddTitlePageDataLine{Urabstimmung vom}{16. April 2013}
\AddTitlePageDataLine{Vorangegangene Fassung vom}{12. Dezember 2018}
\AddTitlePageDataLine{Vom STUPA angenommen am}{\lastChanged}
\AddTitlePageDataLine{Vom Präsidium bestätigt am}{\acceptedUniversity}

\newcommand{\titlepageabstract}{%
  \section*{Rechtsgrundlage}
  Auf Grund von § 65a Abs. 1 S. 1 des Gesetzes über die Hochschulen in Baden-Württemberg
  (Landeshochschulgesetz – LHG) vom 1. Januar 2005, zuletzt geändert am 13. März 2018 durch
  Art. 2 des Gesetzes zur Einführung einer Verfassten Studierendenschaft und zur Stärkung der
  akademischen Weiterbildung (Verfasste-Studierendenschafts-Gesetz – VerfStudG) hat die
  Studierendenschaft der Hochschule Reutlingen in der Urabstimmung vom 16. April 2013, zuletzt
  geändert am \lastChanged~durch Beschluss des Studierendenparlamentes, die nachfolgende
  Organisationssatzung beschlossen.

  Die Hochschule Reutlingen hat die Organisationssatzung durch Beschluss des Präsidiums am \acceptedUniversity~genehmigt.

  \section*{Anlass der Änderung}
  Die Änderung vom \lastChanged~hat den primären Zweck, die Organisationssatzung an die infolge der neuen Fakultät
  „NXT“ geänderten Vertreter:innen-Anzahl anzupassen. Darüber hinaus wurden mehrere Fehler behoben und die Formulierung vereinfacht.
  Es wurde die Möglichkeit geschaffen, Referatsleitungen für ihr Engagement zu entschädigen.
  Die Rolle der Fachschaften wurde genauer definiert.
  Den Referaten und Fachschaften wird eine umfangreichere finanzielle Selbstbestimmung ermöglicht.
  Es wird eine Stimmrechtsübertragung ermöglicht.

  \section*{Präambel}
  Im Folgenden wird aus Gründen der besseren Lesbarkeit ausschließlich die männliche Form
  verwendet. Es können alle Amts-, Status- und Funktionsbezeichnungen, die in dieser Ordnung in
  der männlichen Sprachform verwendet werden, in der entsprechenden weiblichen Sprach-form
  geführt werden.
}

% ==============================================================================
% End of General Settings
% ==============================================================================

% ------------------------------------------------------------------------------
% Logo Configuration Syntax
% ------------------------------------------------------------------------------
% \AddLogo{LogoName}{Scale}
%
% Parameters:
%   LogoName - Name of the logo
%   Scale   - Scaling factor
%
% Logo files must be placed in: HSRTReport/Assets/Images/
% ------------------------------------------------------------------------------

% ------------------------------------------------------------------------------
% Available Logos
% ------------------------------------------------------------------------------
% Below are pre-configured logos for various departments and organizations.
% Uncomment (remove the % at the beginning) the logo(s) you want to use.
% Multiple logos can be active simultaneously and will be arranged automatically.
% ------------------------------------------------------------------------------

% --- INF/Kombiniert ---
% \AddLogo{INF/Kombiniert}{0.9}

% --- INF/Simple ---
% \AddLogo{INF/Simple}{0.9}

% --- STUPA ---
\AddLogo{STUPA}{0.9}

% --- HSRT ---
% \AddLogo{HSRT-Grau}{0.9}

% ==============================================================================
% End of Logo Configuration
% ==============================================================================
