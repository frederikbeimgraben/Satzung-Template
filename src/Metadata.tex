% !TEX root = ../Main.tex
% ==============================================================================
% Document-Specific General Settings
% ==============================================================================
% Description: This file contains all document-specific settings including
%              title, author information, abstract, keywords, and other
%              metadata required for the title page and document properties.
% Author: Frederik Beimgraben
% Date: 23.11.2025
% ==============================================================================

% ------------------------------------------------------------------------------
% Watermark and Date Configuration
% ------------------------------------------------------------------------------
% Configure document watermark and creation date
% These settings affect the document's visual presentation and metadata
% ------------------------------------------------------------------------------

% Watermark text (leave empty for no watermark)
% Example: \newcommand{\waterMarkText}{DRAFT}
% Example: \newcommand{\waterMarkText}{CONFIDENTIAL}
\newcommand{\waterMarkText}{}

% Document creation date
% Format: DD.MM.YYYY
% This date appears on the title page and in document metadata
\createdon{XX.XX.20XX}

% ------------------------------------------------------------------------------
% Document Title
% ------------------------------------------------------------------------------
% Main title of the document
% This appears on the title page and in PDF metadata
% For multi-line titles, use \\ for line breaks
% ------------------------------------------------------------------------------
\title{Template für wissenschaftliche Arbeiten}

% ------------------------------------------------------------------------------
% Title Page Data Fields
% ------------------------------------------------------------------------------
% Configure all fields that appear on the title page
% Use \AddTitlePageDataLine{Label}{Content} for each field
% Use \AddTitlePageDataSpace{dimension} to add vertical spacing
% Use \newline within content for line breaks
% ------------------------------------------------------------------------------

% --- Document Topic – MeTI-SAT Specific ---
% The main topic or theme of the document
% \AddTitlePageDataLine{Thema}{
%	 Thema-XXX: \newline
%	 Template für wissenschaftliche Arbeiten
% }

\AddTitlePageDataSpace{5pt}


% --- Author Information ---
% Student name, semester, and contact information
\author{Hans Maria Muster}
\makeatletter
\AddTitlePageDataLine{Vorgelegt von}{
	\@author \newline
	X. Fachsemester \newline
	\href{mailto:hans-maria.muster@student.hs-reutlingen.de}{hans-maria.muster@student.hs-reutlingen.de}
}
\makeatother

\AddTitlePageDataSpace{5pt}

% --- Submission Date ---
% Date when the document is submitted
% Format: DD.MM.YYYY
\AddTitlePageDataLine{Vorgelegt am}{XX.XX.20XX}

\AddTitlePageDataSpace{5pt}

% --- Academic Information ---
% Study program and course details
\AddTitlePageDataLine{Studiengang}{Medizinisch Technische Informatik B.Sc.}

% Module/Course information
\AddTitlePageDataLine{Modul}{
	METIX.X \newline
	Mustermodul
}

% --- Supervisor Information ---
% Professor or supervisor name
\AddTitlePageDataLine{Dozent:in}{Prof. Dr. Max Mustermann}

% --- Semester Information ---
% Academic semester (e.g., Wintersemester 2024/2025)
\AddTitlePageDataLine{Semester}{Wintersemester 20XX/20XX}

% --- Word Count – MeTI-SAT Specific ---
% Automatic word count for the main content
% Note: This only counts words in the specified file(s) (e.g. content.tex)
% For multiple files, use \quickwordcount{file1,file2,file3}
% \AddTitlePageDataLine{Wortanzahl}{\quickwordcount{content}}

% ------------------------------------------------------------------------------
% Abstract
% ------------------------------------------------------------------------------
% The abstract provides a concise summary of the document
% It should include:
%   - Research objective/hypothesis
%   - Methodology used
%   - Key findings/results
%   - Main conclusions
% Keep it brief (typically 150-250 words)
% ------------------------------------------------------------------------------
\newcommand{\titlepageabstract}{%
	Das Abstract beschreibt in wenigen Sätzen die Zielsetzung und das Ergebnis
	der Ausarbeitung. Das Abstract muss sich vollständig auf der Titelseite
	befinden. Die Zeichensatzformatierung wird in einem eigenen Absatz
	beschrieben. Das Abstract soll es den Lesern:innen ermöglichen, innerhalb
	von wenigen Augenblicken zu erfassen, welcher Inhalt hinter der Überschrift
	steckt und ob das Thema, aus Sicht der Leser:innen, zur weiteren Bearbeitung
	lohnt. Das Abstract ist keine verbale Beschreibung des Inhaltsverzeichnisses,
	sondern gibt kurz und knapp z.B. die Zielsetzung (z.B. Hypothese), die
	eingesetzten Methoden und die erzielten Ergebnisse / Erkenntnisse bekannt.
	Weitere Hinweise finden Sie außerdem im Vorlesungsskript.
}

% ------------------------------------------------------------------------------
% Keywords
% ------------------------------------------------------------------------------
% Keywords help categorize and index the document
% Separate keywords with commas
% Choose 3-7 relevant keywords that describe the main topics
% Use a standard selection of keywords (e.g. ACM CCS oder IEEE) to allow for
% easy categorization and indexing of the document.
% ------------------------------------------------------------------------------
\newcommand{\titlepagekeywords}{%
	Seminararbeit, wissenschaftliche Ausarbeitung, Bachelor-Thesis, Studium, Plagiat
}

% ------------------------------------------------------------------------------
% Module Name
% ------------------------------------------------------------------------------
% The module name for this document (appears in footer)
% This is typically the course or module code and name
% ------------------------------------------------------------------------------
\newcommand{\modulename}{METIX.Y – Mustermodul – WiSe XX/YY}

% ==============================================================================
% End of General Settings
% ==============================================================================

% ------------------------------------------------------------------------------
% Logo Configuration Syntax
% ------------------------------------------------------------------------------
% \AddLogo{LogoName}{Scale}
%
% Parameters:
%   LogoName - Name of the logo
%   Scale   - Scaling factor
%
% Logo files must be placed in: HSRTReport/Assets/Images/
% ------------------------------------------------------------------------------

% ------------------------------------------------------------------------------
% Available Logos
% ------------------------------------------------------------------------------
% Below are pre-configured logos for various departments and organizations.
% Uncomment (remove the % at the beginning) the logo(s) you want to use.
% Multiple logos can be active simultaneously and will be arranged automatically.
% ------------------------------------------------------------------------------

% --- INF/Kombiniert ---
\AddLogo{INF/Kombiniert}{0.9}

% --- INF/Simple ---
% \AddLogo{INF/Simple}{0.9}

% --- HSRT ---
% \AddLogo{HSRT}{0.9}

% ==============================================================================
% End of Logo Configuration
% ==============================================================================
