% !TEX root = Main.tex
% ==============================================================================
% Glossary and Abkürzung Definitions
% ==============================================================================
% Description: This file contains all glossary entries and acronyms used in
%              the document. Entries are automatically sorted and formatted.
% Usage:       Reference entries in text using:
%              - \gls{label} for glossary entries
%              - \acrshort{label} for abbreviation short form
%              - \acrlong{label} for abbreviation long form
%              - \acrfull{label} for full abbreviation (short and long)
% Author: Frederik Beimgraben
% Date: 19.11.2025
% ==============================================================================

% ------------------------------------------------------------------------------
% Glossary Entries
% ------------------------------------------------------------------------------
% Define technical terms, concepts, and important words
% Syntax: \newglossaryentry{label}{
%     name={Display Name},
%     description={Detailed description},
%     plural={Plural form (optional)},
%     genitive={Genitive form (optional)}
% }
% ------------------------------------------------------------------------------

% Example: Technical term
\newglossaryentry{Textkörper}
{
	name=Textkörper,
	description={
			Bezeichnung für den Bereich innerhalb der Arbeit, in dem die
			eigentliche Ausarbeitung enthalten ist. Der Textkörper umfasst
			alle Kapitel zwischen Einleitung und Fazit.
		},
	genitive=Textkörpers,
	plural=Textkörper
}

% ------------------------------------------------------------------------------
% Add more glossary entries here
% ------------------------------------------------------------------------------
% \newglossaryentry{example}
% {
%     name=Example,
%     description={An example glossary entry for demonstration purposes}
% }

% ==============================================================================
% Abkürzung Definitions
% ==============================================================================
% Define abbreviations that appear in the document
% Syntax: \newacronym{label}{SHORT}{Long Form}
% ------------------------------------------------------------------------------

% Document-related abbreviations
\newacronym{a:Abb}{Abb.}{Abbildung}
\newacronym{a:Tab}{Tab.}{Tabelle}

% Legal and regulatory abbreviations
\newacronym{MPG}{MPG}{Medizinproduktegesetz}

% Company and organization abbreviations
\newacronym{a:MS}{MS}{Microsoft®}

% ------------------------------------------------------------------------------
% Add more abbreviations here (alphabetically sorted recommended)
% ------------------------------------------------------------------------------
% Technical abbreviations
% \newacronym{a:API}{API}{Application Programming Interface}
% \newacronym{a:CPU}{CPU}{Central Processing Unit}
% \newacronym{a:GPU}{GPU}{Graphics Processing Unit}

% Academic abbreviations
% \newacronym{a:BSc}{B.Sc.}{Bachelor of Science}
% \newacronym{a:MSc}{M.Sc.}{Master of Science}
% \newacronym{a:PhD}{Ph.D.}{Doctor of Philosophy}

% ==============================================================================
% Glossary Configuration Notes
% ==============================================================================
% - Entries are automatically sorted alphabetically
% - Unused entries can be included using \glsaddallunused in Main.tex
% - Multiple glossary types can be defined if needed
% - Cross-references between entries are supported
% ==============================================================================

% ==============================================================================
% End of Glossary Definitions
% ==============================================================================
