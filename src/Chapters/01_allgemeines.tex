% !TEX root = ../Main.tex
% ==============================================================================
% Erster Abschnitt: Allgemeine Bestimmungen
% ==============================================================================

\chapter{Allgemeine Bestimmungen}
\vspace*{-1.5em}
\rule{\textwidth}{0.5mm}
\vspace*{-3em}

\setcounter{section}{1}
\section*{Erster Unterabschnitt: Rechtsstellung}
\addcontentsline{toc}{section}{\textbf{Rechtsstellung}}
\sectionmark{Rechtsstellung}
\label{sec:allg:rechtsstellung}

\subsection{Rechtsstellung}
\label{subsec:allg:rechtsstellung}
\label{subsec:para:1}
Die immatrikulierten Studierenden (Studierende) der Hochschule Reutlingen bilden die Verfasste Studierendenschaft (Studierendenschaft). Sie ist eine rechtsfähige Körperschaft des öffentlichen Rechts und als solche eine Gliedkörperschaft der Hochschule. Sie nimmt ihre Angelegenheiten im Rahmen der gesetzlichen Bestimmungen selbstständig wahr und untersteht der Rechtsaufsicht des Präsidiums der Hochschule. Sie führt den Namen „Verfasste Studierendenschaft der Hochschule Reutlingen“. Ihr Sitz ist Reutlingen. Nachfolgend wird sie nur Studierendenschaft genannt.

\subsection{Aufgaben}
\label{subsec:allg:aufgaben}
\label{subsec:para:2}
\begin{enumerate}[label=(\arabic*)]
  \item Die Studierendenschaft hat die Aufgabe, die Interessen der Studierenden wahrzunehmen. Sie hat unbeschadet der Zuständigkeit der Hochschule und des Studentenwerks die folgenden Aufgaben:
  \label{subsec:allg:aufgaben:abs:1}
  \begin{enumerate}[label=\alph*)]
    \item die Wahrnehmung der hochschulpolitischen, fachlichen und fachübergreifenden sowie der sozialen, wirtschaftlichen und kulturellen Belange der Studierenden,
    \item die Mitwirkung an den Aufgaben der Hochschulen nach den §§ 2 bis 7 LHG,
    \item die Förderung der politischen Bildung und des staatsbürgerlichen Verantwortungsbewusstseins der Studierenden,
    \item die Förderung der Gleichstellung und den Abbau von Benachteiligungen innerhalb der Studierendenschaft,
    \item die Förderung der sportlichen Aktivitäten der Studierenden,
    \item die Pflege der überregionalen und internationalen Studierendenbeziehungen.
  \end{enumerate}

  \item Zur Erfüllung ihrer Aufgaben ermöglicht die Studierendenschaft den Meinungsaustausch in der Gruppe der Studierenden und kann insbesondere auch zu solchen Fragen Stellung beziehen, die sich mit der gesellschaftlichen Aufgabenstellung der Hochschule, ihrem Beitrag zur nachhaltigen Entwicklung sowie mit der Anwendung der wissenschaftlichen Erkenntnisse und der Abschätzung ihrer Folgen für die Gesellschaft und die Natur beschäftigen.

  \item Im Rahmen der Erfüllung ihrer Aufgaben nach \hyperref[subsec:allg:aufgaben:abs:1]{Absatz 1} nimmt die Studierendenschaft ein politisches Mandat wahr. Sie wahrt nach den verfassungsrechtlichen Grundsätzen die weltanschauliche, religiöse und parteipolitische Neutralität.

  \item Beabsichtigt die Studierendenschaft, nicht nur vorübergehend konkrete Aufgaben oder Angebote innerhalb ihrer Zuständigkeit wahrzunehmen, die bereits von dem für die Hochschule zuständigen Studentenwerk wahrgenommen werden, so gelten die Bestimmungen §65 Abs. 5 LHG.
\end{enumerate}

\subsection{Zentrale Organe der Studierendenschaft}
\label{subsec:allg:zentrale_organe}
\label{subsec:para:3}
Zentrale Organe der Studierendenschaft sind das Studierendenparlament (STUPA) und der Allgemeine Studierendenausschuss (AStA). Das STUPA entscheidet über die grundsätzlichen Angelegenheiten der Studierendenschaft einschließlich der Satzungen (legislatives Organ). Die laufenden Geschäfte werden vom AStA geführt (exekutives Organ); der Vorsitzende des AStA vertritt die Studierendenschaft nach innen und nach außen; seine Stellvertreter vertreten ihn.

\subsection{Dezentrale Gliederung der Studierendenschaft in Fakultätsräte und Fachschaften}
\label{subsec:allg:dezentrale_gliederung}
\label{subsec:para:4}
Auf dezentraler Ebene gliedert die Studierendenschaft sich in Fakultätsräte und Fachschaften. Einer Fachschaft gehören alle Studierenden einer Fakultät der Hochschule an. Die Fakultätszugehörigkeit richtet sich nach § 22 Absatz 3 LHG. Jede Fachschaft wählt nach \cref{subsec:para:10} Absatz 1 Fachschaftsvertretungen. Darüber hinaus werden nach \cref{subsec:para:10} Absatz 2 studentische Fakultätsräte einer jeder Fakultät gewählt.

\subsection{Mitgliedschaft und Mitwirkung in Gremien}
\label{subsec:allg:mitgliedschaft_mitwirkung}
\label{subsec:para:5}
\begin{enumerate}[label=(\arabic*)]
\item Die Mitglieder der Studierendenschaft haben das Recht und die Pflicht, an der Selbstverwaltung und der Erfüllung der Aufgaben der Studierendenschaft in Organen, Gremien und beratenden Ausschüssen mit besonderen Aufgaben mitzuwirken und Ämter, Funktionen und sonstige Pflichten in der Selbstverwaltung zu übernehmen, es sei denn, dass wichtige Gründe entgegenstehen. Wer ein Amt, eine Wahlmitgliedschaft in einem Gremium oder eine sonstige gesetzliche oder in dieser Satzung vorgesehene Funktion übernommen hat, muss diese nach einer Beendigung bis zum Amtsantritt eines Nachfolgers kommissarisch fortführen.

\item Die Mitglieder in den Organen der Studierendenschaft üben ihre Tätigkeit ehrenamtlich aus. \cref{subsec:vermoegen:arbeitsentgelte-aufwandsentschaedigungen} Absatz 2 bleibt unberührt.

\item Wer eine Tätigkeit in der Selbstverwaltung übernommen hat, muss die ihm übertragenen Geschäfte uneigennützig und verantwortungsbewusst führen. Mitglieder von Gremien sind zur Verschwiegenheit verpflichtet über alle Angelegenheiten und Tatsachen, die ihnen in Personal- und Prüfungsangelegenheiten in nicht-öffentlicher Sitzung bekannt geworden sind. Diese Verpflichtungen gelten auch nach Beendigung der Tätigkeit und schließen die Beratungsunterlagen ein.

\item Studierende, die vorsätzlich oder grob fahrlässig die ihnen obliegenden Pflichten verletzen, insbesondere Gelder der Studierendenschaft für die Erfüllung anderer als der hochschulgesetzlich zulässigen Aufgaben verwenden, haben der Studierendenschaft den ihr daraus entstehenden Schaden zu ersetzen. Für die Verjährung von Ansprüchen der Studierendenschaft gelten § 59 LBG und § 48 BeamtStG entsprechend.

\item Mitglieder in den Organen der Studierendenschaft werden wegen ihrer Tätigkeit in der Studierendenschaft nicht benachteiligt. Eine Tätigkeit als gewähltes Mitglied in gesetzlich vorgesehenen Gremien oder satzungsmäßigen Organen der Studierendenschaft während mindestens eines Jahres kann entsprechend bei der Berechnung der Prüfungsfristen unberücksichtigt bleiben; die Entscheidung darüber trifft der Präsident der Hochschule.
\end{enumerate}

\subsection{Zusammenwirken mit der Hochschule}
\label{subsec:allg:zusammenwirken}
\label{subsec:para:6}
Die Studierendenschaft und ihre Trägerkörperschaft, die Hochschule, verfolgen gemeinsame Interessen. Die Studierendenschaft strebt eine intensive Zusammenarbeit mit der Hochschule an und informiert die Hochschule frühzeitig über ihre Planungen. Die Studierendenschaft wird regelmäßige Gesprächstermine mit dem Hochschulpräsidium zum gemeinsamen Informationsaustausch wahrnehmen.

\setcounter{section}{2}
\section*{Zweiter Unterabschnitt: Allgemeine Verfahrensvorschriften für Gremien}
\addcontentsline{toc}{section}{\textbf{Allgemeine Verfahrensvorschriften für Gremien}}
\sectionmark{Verfahrensvorschriften}
\label{sec:allg:verfahrensvorschriften}

\subsection{Hochschulöffentlichkeit}
\label{subsec:allg:hochschuloeffentlichkeit}
\label{subsec:para:7}
Die Sitzungen des Studierendenparlamentes, des Allgemeinen Studierendenausschusses und der Fachschaftsvertretungen sind hochschulöffentlich. Abweichend von Satz 1 werden Personal- und Prüfungsangelegenheiten in nicht-öffentlicher Sitzung behandelt. Die Hochschulöffentlichkeit kann darüber hinaus durch Beschluss für die gesamte Sitzung oder für einzelne Tagesordnungspunkte ausgeschlossen werden; in diesem Fall ist das Ergebnis der Sitzung in geeigneter Weise bekannt zu machen.

\subsection{Beschlussfähigkeit}
\label{subsec:allg:beschlussfaehigkeit}
\label{subsec:para:8}
\begin{enumerate}[label=(\arabic*)]
\item Ein Gremium der Studierendenschaft ist beschlussfähig, wenn mindestens 40\% aller stimmberechtigten Mitglieder anwesend sind und die Sitzung ordnungsgemäß einberufen wurde.

\item Ist ein Gremium nicht beschlussfähig, so ist eine weitere Sitzung des Gremiums mit derselben Tagesordnung zu berufen. Zwischen den beiden Sitzungen sollen mindestens zwei Werktage liegen. Das Gremium ist in der Wiederholungssitzung beschlussfähig, wenn mindestens 25\% aller stimmberechtigten Mitglieder anwesend sind und in der Einladung auf die erleichterte Beschlussfähigkeit hingewiesen wurde.
\end{enumerate}

\subsection{Beschlussfassung und Bekanntgabe von Beschlüssen}
\label{subsec:allg:beschlussfassung}
\label{subsec:para:9}
\begin{enumerate}[label=(\arabic*)]
\item Soweit in dieser Satzung keine anderweitige Regelung getroffen worden ist, kommen Beschlüsse mit der \hl{einfachen} Mehrheit der abgegebenen Stimmen der anwesenden stimmberechtigten Mitglieder zustande; Stimmenthaltungen und ungültige Stimmen gelten als nicht abgegebene Stimmen. \hl{Eine Stimmrechtsübertragung ist dem Vorsitzenden des Gremiums spätestens 2 Stunden vor Beginn der Sitzung mitzuteilen und nur auf ein anderes Mitglied des Gremiums möglich. Sofern noch kein Vorsitzender bestimmt ist, ist keine Stimmrechtsübertragung möglich.}

\item Sofern diese Satzung keine besonderen Bestimmungen enthält, werden Beschlüsse der zentralen Gremien der Studierendenschaft durch Aushang an der Anschlagtafel „Amtliche Mitteilungen der Studierendenschaft an der Hochschule Reutlingen“ bekanntgemacht. Die Aushangfrist beträgt zehn Werktage. Der Samstag ist kein Werktag im Sinne dieser Satzung. Der Tag des Beginns und der Beendigung des Aushangs ist auf dem Beschluss zu vermerken. Außerdem werden die Studierenden per Mail über den Beschluss informiert.

\item Satzungen der Studierendenschaft werden vom Präsidium der Hochschule in der für Hochschulsatzungen vorgesehenen Weise als Satzungen der Gliedkörperschaft bekanntgemacht.
\end{enumerate}

\subsection{Wahlen zu den Gremien und Wahlen in Gremien}
\label{subsec:allg:wahlen}
\label{subsec:para:10}
\begin{enumerate}[label=(\arabic*)]
\item Die direkt gewählten Mitglieder des Studierendenparlamentes, sowie die studentischen Fakultätsräte werden nach Maßgabe des Hochschulgesetzes in allgemeiner, freier, gleicher, unmittelbarer und geheimer Wahl nach den Grundsätzen der Mehrheitswahl mit Bindung an den Listenvorschlag gewählt. Die Studierenden der Hochschule haben das aktive und passive Wahlrecht.

\item Die studentischen Fakultätsräte wählen aus ihrer Mitte zwei ständige Mitglieder und zwei Ersatzmitglieder für das Studierendenparlament in freier, gleicher und geheimer Wahl.

\item Im Übrigen wird bei Wahlen in den Gremien der Studierendenschaft, wenn niemand widerspricht, durch Zeichen gewählt. Auf Verlangen eines Stimmberechtigten ist geheim zu wählen.

\item Bei Personenwahlen mit mindestens zwei Kandidaten erfolgt eine geheime Abstimmung.

\item Die Amtszeit der Mitglieder der zentralen Organe und sonstigen Gremien einschließlich der Fachschaftsvertretungen beträgt ein Jahr. Sie beginnt am 1. Oktober und endet am 30. September des Folgejahres. Bei einer unterjährigen Wahl oder Nachwahl wird die Amtszeit \hl{auf die bis zum letzten Tag des Sommersemesters verbleibende Zeit verkürzt.}

\item Die Studierendenschaft erlässt eine Wahlsatzung, in der insbesondere die Abstimmung, die Ermittlung des Wahlergebnisses, die Wahlprüfung sowie die weiteren Einzelheiten des Wahlverfahrens einschließlich Briefwahl und Online-Wahl geregelt werden. Die Wahlsatzung soll Regelungen treffen, welche schriftlichen Erklärungen in Wahlangelegenheiten durch einfache elektronische Übermittlung, durch mobile Medien oder in elektronischer Form abgegeben werden können.
\end{enumerate}

\subsection{Geschäftsordnung}
\label{subsec:allg:go}
\label{subsec:para:11}
Das Studierendenparlament, der Allgemeine Studierendenausschuss und die Fachschaftsvertretungen regeln ihren Geschäftsgang durch Geschäftsordnungen (GO). Solange eine Fachschaft keine eigene GO beschließt, \hl{gilt entsprechend die GO des STUPA. Siehe dazu} \cref{subsec:fachschaft:satzung} Absatz 2.
