% !TEX root = ../Main.tex
% ==============================================================================
% Vierter Abschnitt: Studierendenbefragung und Vollversammlung
% ==============================================================================

\newpage
\chapter{Studierendenbefragung und Vollversammlung}
\sectionmark{}
\vspace*{-1.5em}
\rule{\textwidth}{0.5mm}
\vspace*{-2em}

\subsection{Zweck}
\label{subsec:vollversammlung:zweck}
Innerhalb der Studierendenschaft können Studierendenbefragungen oder Vollversammlungen der Studierendenschaft zu Belangen nach \cref{subsec:allg:aufgaben} durchgeführt werden, die der Meinungsbildung dienen.

\subsection{Vollversammlung}
\label{subsec:vollversammlung}
\begin{enumerate}[label=(\arabic*)]
  \item Die Vollversammlung (VV) ist die studentische Versammlung aller an der Hochschule immatrikulierten Studierenden.
  \item Angelegenheit der VV ist die Beratung aller Belange, welche die Studierenden der Hochschule betreffen. Die VV dient der Information aller Studierenden und kann unverbindliche Empfehlungen an die Organe der verfassten Studierendenschaft erarbeiten.
  \item Die VV kann vom AStA mit einer Frist von mindestens 2 Wochen einberufen werden. Sie muss einberufen werden:
  \begin{enumerate}[label=\alph*)]
    \item auf Antrag von 5 v. Hundert der immatrikulierten Studierenden der Hochschule,
    \item auf Antrag einer Zweidrittelmehrheit der Mitglieder des Studierendenparlamentes,
    \item wenn dies mindestens ein Drittel der gewählten Fachschaftsvertretungen verlangen oder
    \item auf Antrag einer Zweidrittelmehrheit der Mitglieder des AStA.
  \end{enumerate}
  \item Der Haushaltsplan, die Wahl von Gremienvertretern und -vertreterinnen, Satzungen und Ordnungen können nicht Gegenstand einer Empfehlung der Vollversammlung sein.
  \item Die Einberufung und Durchführung einer VV obliegt dem AStA.
\end{enumerate}

\subsection{Studierendenbefragung}
\label{subsec:studierendenbefragung}
\begin{enumerate}[label=(\arabic*)]
  \item Eine Studierendenbefragung findet statt,
  \begin{enumerate}[label=\alph*)]
    \item auf Antrag von 5 v. Hundert der immatrikulierten Studierenden der Hochschule,
    \item auf Antrag einer Zweidrittelmehrheit der Mitglieder des Studierendenparlamentes,
    \item wenn dies mindestens ein Drittel der gewählten Fachschaftsvertretungen verlangen oder
    \item auf Antrag einer Zweidrittelmehrheit der Mitglieder des AStA.
  \end{enumerate}
  \item Das Ergebnis der Studierendenbefragung hat empfehlenden Charakter für das Studierendenparlament. Das Studierendenparlament muss sich bei seiner nächsten stattfindenden Sitzung, frühestens jedoch zwei Wochen nach Bekanntgabe des Abstimmungsergebnisses, mit diesem auseinandersetzen.
  \item Der Haushaltsplan, die Wahl von Gremienvertretern und -vertreterinnen, Satzungen und die Ordnungen können nicht Gegenstand von Studierendenbefragungen sein.
  \item Jede Studierendenbefragung wird von mindestens einer Veranstaltung zum Zwecke der Information und Diskussion der zur Abstimmung stehenden Fragen begleitet. Zwischen Informationsveranstaltung und Beginn der Studierendenbefragung dürfen nicht mehr als zwei Wochen liegen.
  \item Die Einberufung und Durchführung einer Studierendenbefragung obliegt dem AStA.
\end{enumerate}
