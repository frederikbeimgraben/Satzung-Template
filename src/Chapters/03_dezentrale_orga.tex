% !TEX root = ../Main.tex
% ==============================================================================
% Dritter Abschnitt: Dezentrale Organisation
% ==============================================================================

\newpage
\chapter{Dezentrale Organisation}
\sectionmark{Fachschaft und Fachschaftsvertretung}
\vspace*{-1.5em}
\rule{\textwidth}{0.5mm}
\vspace*{-2em}

\subsection{Fachschaft und Fachschaftsvertretung}
\label{subsec:fachschaft}
Die Studierenden einer Fakultät bilden eine Fachschaft. In der Fachschaft wird eine Fachschaftsvertretung gebildet. Die Fachschaftsvertretung nimmt die fakultätsbezogenen Studienangelegenheiten und Aufgaben im Sinne des § 65 Absatz 2 LHG auf Fakultätsebene wahr.

\subsection{\hl{Finanzen der Fachschaft}}
\label{subsec:fachschaft:finanzen}
\begin{enumerate}[label=(\arabic*)]
  \item \hl{Falls das STUPA den jeweiligen Fachschaften ein Budget für das Haushaltsjahr zur Verfügung stellt, dürfen die Fachschaftsvertretungen über das der Fachschaft zugeteilte Budget, verfügen. Dies muss jedoch in Absprache mit ihren jeweiligen studentischen Fakultätsräten, dem Haushaltsbeauftragten der Studierendenschaft, sowie dem Finanzreferenten des AStA erfolgen.}
  \item \hl{Bei unsachgemäßem Umgang mit dem Budget muss der AStA oder das STUPA die zur Verfügungstellung des Budgets überprüfen und kann der jeweiligen Fachschaftsvertretung die Berechtigung zur eigenständigen Verfügung über das Budget entziehen. In diesem Fall ist die Entscheidung des STUPA oder AStA gegenüber der Fachschaftsvertretung schriftlich zu begründen.}
\end{enumerate}

\subsection{\hl{Satzung und Geschäftsordnung der Fachschaft}}
\label{subsec:fachschaft:satzung}
\begin{enumerate}[label=(\arabic*)]
  \item \hl{Eine Fachschaftsvertretung kann eine Satzung beschließen, um ihre interne Organisation zu regeln. Diese ergänzt die Bestimmungen der Organisationssatzung der Studierendenschaft, darf jedoch nicht mit ihr im Konflikt stehen. Von Fachschaftsvertretungen beschlossene Satzungen müssen vom Studierendenparlament genehmigt werden.}
  \item \hl{Eine Fachschaftsvertretung kann eine Geschäftsordnung beschließen, um ihren Geschäftsgang zu regeln. Sofern eine Fachschaftsvertretung keine Geschäftsordnung beschlossen hat, gilt entsprechend die Geschäftsordnung des Studierendenparlamentes. Der AStA ist über verabschiedete Geschäftsordnungen durch die Fachschaftsvertretungen in Kenntnis zu setzen.}
\end{enumerate}

\subsection{Zusammensetzung der Fachschaftsvertretung}
\label{subsec:fachschaft:zusammensetzung}
Die Fachschaftsvertretung setzt sich aus den gewählten Fachschaftsvertretungen sowie den studentischen Fakultätsratsmitgliedern, die der Fachschaftsvertretung kraft \hl{ihres} Amtes angehören, zusammen.

\subsection{Wahlen zu den Fachschaftsvertretungen}
\label{subsec:fachschaft:wahlen}
\begin{enumerate}[label=(\arabic*)]
  \item \hl{Nach Maßgabe von} \cref{subsec:allg:wahlen} Absatz 1 werden studentische Fakultätsräte einer jeden Fakultät gewählt.
  \item \hl{Sofern nicht anders in der Satzung der Fachschaft geregelt, ist die Fachschaftsvertretung gleich der studentischen Mitglieder im jeweiligen Fakultätsrat. Näheres regeln die Satzung und Geschäftsordnung der Fachschaft gemäß} \cref{subsec:fachschaft:satzung}.
\end{enumerate}

\subsection{Fachschaftsvorsitzender}
\label{subsec:fachschaft:fachschaftsvorsitzender}
\begin{enumerate}[label=(\arabic*)]
  \item Der Fachschaftsvorsitzende führt die laufenden Geschäfte der Fachschaft, bereitet die Beschlüsse der Fachschaftsvertretung vor und führt sie aus.
  \item Er wird von der Fachschaftsvertretung aus ihrer Mitte für die Dauer der Amtszeit gewählt. Für die Wahl ist die Mehrheit der Stimmen der anwesenden Mitglieder erforderlich. Wird diese Mehrheit in zwei Wahlgängen nicht erreicht, so ist gewählt, wer im dritten Wahlgang die meisten Stimmen erhalten hat.
  \item Der Fachschaftsvorsitzende verliert das Amt vor Ablauf der Amtszeit durch Neuwahl eines Fachschaftsvorsitzenden mit der Mehrheit der Stimmen der anwesenden Mitglieder der Fachschaftsvertretung, durch Ausscheiden aus der Fachschaftsvertretung oder durch Rücktritt aus wichtigem Grund. Der Rücktritt ist schriftlich gegenüber den anderen Mitgliedern der Fachschaftsvertretung zu erklären.
\end{enumerate}

\subsection{Konstituierende Sitzung}
\label{subsec:fachschaft:konstituierende_sitzung}
Die erste Fachschaftsvertretungssitzung der jeweiligen Amtsperiode wird von dem mit den höchsten Stimmzahlen gewählten Mitglied der Fachschaftsvertretung unverzüglich nach Beginn der Amtszeit einberufen. Dieses Mitglied leitet die Sitzung, bis die Wahl des Fachschaftsvorsitzenden abgeschlossen ist.
