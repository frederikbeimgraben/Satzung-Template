% !TEX root = ../Main.tex
% ==============================================================================
% Sechster Abschnitt: Schlussbestimmungen
% ==============================================================================

\newpage
\chapter{Schlussbestimmungen}
\sectionmark{}
\vspace*{-1.5em}
\rule{\textwidth}{0.5mm}
\vspace*{-2em}

\subsection{Änderung der Organisationssatzung}
\label{subsec:org-satzung:aenderung}
\begin{enumerate}[label=(\arabic*)]
  \item Die Organisationssatzung kann durch Änderungssatzung, die mit einer Mehrheit von zwei Dritteln der Mitglieder des Studierendenparlamentes beschlossen werden muss, geändert werden. Die Änderungssatzung muss vom Präsidium der Hochschule genehmigt und in der für Hochschulsatzungen vorgesehenen Weise bekannt gemacht werden.
  \item Die Organisationssatzung kann auch durch Änderungssatzung, die in einer Urabstimmung unter den Mitgliedern der Studierendenschaft beschlossen wird, geändert werden. Der Beschluss über die Änderungssatzungen zur Organisationssatzung bedarf der Zustimmung von mindestens der Hälfte der an der Abstimmung teilnehmenden Studierenden. Änderungssatzungsvorschläge mit Erläuterungen sind beim Vorsitzenden der Studierendenschaft einzureichen. Sie müssen dem geltenden Recht entsprechen und von mindestens 2 \% der Studierenden unterzeichnet sein. Der Stichtag für die Bestimmung der Größe der Studierendenschaft ist der jeweils vorangegangene 01. April bzw. 01. November. Das Studierendenparlament legt den Termin für die Urabstimmung fest und informiert entsprechend.
  \label{subsec:org-satzung:aenderung:abs:2}
\end{enumerate}

\subsection{Schlichtungskommission}
\label{subsec:schlichtungskommission}
\begin{enumerate}[label=(\arabic*)]
  \item Zur Beilegung von Streitigkeiten innerhalb der Studierendenschaft wird eine Schlichtungskommission eingerichtet.
  \item Die Schlichtungskommission besteht aus drei Mitgliedern, die vom Studierendenparlament gewählt werden.
  \item Die Schlichtungskommission entscheidet nach freien Ermessen und bemüht sich um eine einvernehmliche Lösung.
  \item Die Entscheidungen der Schlichtungskommission sind für die Beteiligten verbindlich.
\end{enumerate}

\subsection{Konstituierende Wahlen}
\label{subsec:konstituierende-wahlen}
\begin{enumerate}[label=(\arabic*)]
  \item Die erste Wahl des Studierendenparlamentes und des Allgemeinen Studierendenausschusses erfolgt durch Urabstimmung aller Studierenden der Hochschule.
  \item Die Urabstimmung wird vom Wahlvorstand organisiert, der vom vorläufigen Studierendenrat bestellt wird.
  \item Die konstituierende Sitzung findet spätestens zwei Wochen nach Bekanntgabe der Wahlergebnisse statt.
\end{enumerate}

\subsection{Inkrafttreten}
\label{subsec:inkrafttreten}
Diese Satzung tritt mit dem Tag der Genehmigung durch das Präsidium der Hochschule in Kraft.
