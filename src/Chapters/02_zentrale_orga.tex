% !TEX root = ../Main.tex
% ==============================================================================
% Zweiter Abschnitt: Zentrale Organisation
% ==============================================================================

\newpage
\chapter{Zentrale Organisation}
\vspace*{-1.5em}
\rule{\textwidth}{0.5mm}
\vspace*{-3em}

\setcounter{section}{1}
\section*{Erster Unterabschnitt: Das Studierendenparlament (STUPA)}
\addcontentsline{toc}{section}{\textbf{Das Studierendenparlament (STUPA)}}
\sectionmark{Das Studierendenparlament (STUPA)}
\label{sec:stupa}

\subsection{Aufgaben}
\label{subsec:stupa:aufgaben}
Das STUPA entscheidet über die grundsätzlichen Angelegenheiten der Studierendenschaft einschließlich der Satzungen. Es hat insbesondere folgende Aufgaben:
\begin{enumerate}[label=(\arabic*)]
  \item Wahl, Abberufung und Kontrolle der Mitglieder des AStA,
  \item Verabschiedung der Geschäftsordnung des Studierendenparlamentes,
  \item Verabschiedung des Haushaltsplans,
  \item Beratung und Beschlussfassung über alle Satzungen und Ordnungen der Studierendenschaft,
  \item Einsetzen von Referaten und Arbeitskreisen,
  \item Beschlussfassung über Beschwerden von Studierenden, die vorher von der Fachschaft oder dem AStA zurückgewiesen wurden oder direkt bei dem STUPA eingelegt werden,
  \item Beschlussfassung über den Zusammenschluss mit Studierendenschaft anderer Hochschulen zu einem Verband.
\end{enumerate}

\subsection{Zusammensetzung des Studierendenparlamentes}
\label{subsec:stupa:zusammensetzung}
Das Studierendenparlament setzt sich aus insgesamt \hl{bis zu 21} stimmberechtigten Mitgliedern der Studierendenschaft im Sinne von \cref{subsec:allg:aufgaben} zusammen. Fünf Mitglieder werden direkt gewählt. Die studentischen Fakultätsräte jeder Fakultät wählen und entsenden aus ihrer Mitte zwei weitere Mitglieder, \hl{sowie zwei benannte Vertretungen} nach Maßgabe von \cref{subsec:allg:wahlen} Abs. 2. Hinzu treten die vier studentischen Senatsmitglieder als Mitglieder kraft \hl{ihres} Amtes. Darüber hinaus gilt:

\begin{enumerate}[label=(\arabic*)]
  \item Doppelfunktionen sind nicht zulässig.
  \item Jedes Mitglied hat eine Stimme. \cref{subsec:allg:beschlussfassung} Absatz 1 bleibt davon unberührt.
  \item \hl{Die Übertragung der Stimme von einem studentischen Entsandten eines Fakultätsrates auf eine der benannten Vertretungen des selben Fakultätsrates bedarf abweichend von} \cref{subsec:allg:beschlussfassung} Absatz 1 \hl{keiner Mitteilung an den Vorsitzenden des Studierendenparlamentes.}
\end{enumerate}

\subsection{Vorsitzender des Studierendenparlamentes und des Allgemeinen Studierendenausschusses}
\label{subsec:stupa:vorsitzender}
\begin{enumerate}[label=(\arabic*)]
  \item Der Vorsitzende des Studierendenparlamentes wird mit einfacher Mehrheit der Mitglieder des Studierendenparlamentes aus der Mitte des STUPA gewählt. Er ist zugleich Vorsitzender des Allgemeinen Studierendenausschusses.
  \item Der Vorsitzende vertritt die Studierendenschaft nach innen und nach außen.
  \item Der Vorsitzende ist für die Vor- und Nachbereitung sowie die ordnungsgemäße Durchführung der Sitzungen des AStA sowie des STUPA verantwortlich.
  \item Der Vorsitzende wird vom Finanzreferenten des AStA vertreten, wenn er verhindert ist oder sich zeitweilig ablösen lassen muss. Entsprechend vorherigem Satz vertritt der Schriftführer den Finanzreferenten.
  \item Der Vorsitzende wirkt auf die einheitliche Wahrnehmung der Aufgaben der Studierendenschaft hin, koordiniert die Arbeit des AStA und überwacht die Durchführung der Beschlüsse des AStA.
  \item Der Vorsitzende leitet die zentrale Verwaltung der Studierendenschaft und übt die Weisungsbefugnis gegenüber den Bediensteten der Studierendenschaft aus.
  \item Der Vorsitzende erstattet dem Studierendenparlament über die Arbeit des AStA sowie dem AStA über die Arbeit des Studierendenparlamentes Bericht.
\end{enumerate}

\subsection{Ausscheiden von Parlamentsmitgliedern}
\label{subsec:stupa:ausscheiden}
\begin{enumerate}[label=(\arabic*)]
  \item Scheidet ein von den studentischen Fakultätsräten entsandtes Mitglied des Studierendenparlamentes aus, so rückt ein des jeweiligen studentischen Fakultätsrats gewähltes Ersatzmitglied als ständiges Mitglied nach. Die studentischen Fakultätsräte einer jeder Fakultät haben im Weiteren für den Bestand von zwei Ersatzmitgliedern Sorge zu tragen.
  \item Ein von der Fachschaftsvertretung entsandtes Mitglied des Studierendenparlamentes scheidet aus dem STUPA aus
  \begin{enumerate}[label=\alph*)]
    \item mit Ablauf der Amtszeit,
    \item durch Exmatrikulation,
    \item durch Rücktritt aus wichtigem Grund, der dem Vorsitzenden der Studierendenschaft gegenüber schriftlich zu erklären ist,
    \item \hl{durch wiederholtes unentschuldigtes Fehlen bei ordentlichen Sitzungen. Näheres regelt die Geschäftsordung (GO) des STUPAs,}
    \item \hl{durch Tod.}
  \end{enumerate}
  \item Ein Mitglied kraft Amtes (studentisches Senatsmitglied) scheidet aus, wenn es sein Amt als studentisches Senatsmitglied verliert. Der/die Nachfolger/in im Amt rückt in das Studierendenparlament ein.
  \item Scheidet ein direkt gewähltes Mitglied des Studierendenparlamentes aus, so rückt die Person mit der nächsthöheren Stimmenzahl der entsprechenden Mitgliedschaft nach.
  \item Ein direkt gewähltes Mitglied des Studierendenparlamentes scheidet aus dem Parlament aus
  \begin{enumerate}[label=\alph*)]
    \item mit Ablauf der Amtszeit,
    \item durch Exmatrikulation,
    \item durch Rücktritt \hl{aus wichtigem Grund}, der dem Vorsitzenden der Studierendenschaft gegenüber schriftlich zu erklären ist,
    \item \hl{durch wiederholtes unentschuldigtes Fehlen bei ordentlichen Sitzungen. Näheres regelt die Geschäftsordung (GO) des STUPAs,}
    \item durch Tod.
  \end{enumerate}
\end{enumerate}

\subsection{Sitzungen des Studierendenparlamentes}
\label{subsec:stupa:sitzungen}
\begin{enumerate}[label=(\arabic*)]
  \item Zu der ersten Sitzung des Studierendenparlamentes lädt das lebensälteste Mitglied des Studierendenparlamentes ein. Es leitet die Sitzung bis die Wahl zum Vorsitzenden der Studierendenschaft abgeschlossen ist.
  \item Ordentliche Sitzungen des Studierendenparlamentes sollen in der Vorlesungszeit mindestens einmal monatlich abgehalten werden.
  \item Auf Verlangen des Allgemeinen Studierendenausschusses oder auf Verlangen von mindestens 20 \% der Mitglieder des Studierendenparlamentes oder auf Antrag von mindestens 5 \% der gesamten Studierendenschaft finden außerordentliche Sitzungen des Studierendenparlamentes statt.
  \item Dem Schriftführer obliegt die Anfertigung und Veröffentlichung des Protokolls. Bei seiner Verhinderung bestimmt zu Sitzungsbeginn der Vorsitzende einen Protokollführer. Die Niederschrift ist vom Schriftführer zu unterzeichnen und in der nächsten Sitzung des STUPAs zu genehmigen.
\end{enumerate}

\setcounter{section}{2}
\section*{Zweiter Unterabschnitt: Referate des Studierendenparlamentes}
\addcontentsline{toc}{section}{\textbf{Referate des Studierendenparlamentes}}
\sectionmark{Referate des Studierendenparlamentes}
\label{sec:referate}

\subsection{\hl{Referate}}
\label{subsec:stupa:referate}
\begin{enumerate}[label=(\arabic*)]
  \item Das STUPA kann \hl{Referate} einsetzen, die dem Studierendenparlament \hl{gegenüber} für ihre Tätigkeit verantwortlich sind.

  \item \hl{Das STUPA wählt mindestens ein mal im Jahr, auf Forderung von mindestens 20 \% der anwesenden Mitglieder des STUPAs oder auf Forderung des Vorsitzenden, für jedes Referat eine Referatsleitung entsprechend}\relax~\cref{subsec:allg:wahlen}. \hl{Eine Referatsleitung besteht aus einem Vorsitzenden und höchstens vier weiteren Mitgliedern. Alle Mitglieder haben eine Stimme im jeweiligen Referat.}

  \item Den \hl{Referatsleitungen} können \hl{Mitglieder und Nichtmitglieder} des STUPA mit Sitz und Stimme angehören. \hl{Für die Mitgliedschaft gilt entsprechend} \cref{subsec:para:5} Absatz 4. Der Vorsitzende der \hl{Referatsleitung} soll dem Studierendenparlament angehören.

  \item \hl{Mitglieder der Referatsleitung müssen Mitglieder der Studierendenschaft im Sinne von} \cref{subsec:allg:aufgaben} \hl{sein.}

  \item Als ständiges \hl{Referat} wird \hl{das Referat Finanzen} eingerichtet. \hl{Den Vorsitz des Referates übernimmt kraft seines Amtes der Finanzreferent} entsprechend \cref{subsec:asta:zusammensetzung} Absatz 2.

  \item \hl{Referate können selbstständig über die ihnen vom STUPA zugewiesenen Mittel verfügen. Ausgaben bedürfen einer sachlichen Prüfung durch den Haushaltsbeauftragen und den Finanzreferenten. Näheres regelt die Finanzordnung entsprechend} \cref{subsec:vermoegen:haushaltsplan-finanzordnung}.
\end{enumerate}

\newpage
\setcounter{section}{3}
\section*{Dritter Unterabschnitt: Der Allgemeine Studierendenausschuss}
\addcontentsline{toc}{section}{\textbf{Der Allgemeine Studierendenausschuss}}
\sectionmark{Der Allgemeine Studierendenausschuss}
\label{sec:asta}

\subsection{Zusammensetzung des Allgemeinen Studierendenausschusses (AStA)}
\label{subsec:asta:zusammensetzung}
\begin{enumerate}[label=(\arabic*)]
  \item Die Mitglieder des AStA müssen Mitglieder der Studierendenschaft im Sinne von \cref{subsec:allg:aufgaben} sein.
  \item Der AStA setzt sich zusammen aus:
  \begin{enumerate}[label=\arabic*.]
    \item dem Vorsitzenden der Studierendenschaft,
    \item dem Finanzreferenten, der zugleich 1. Stellvertreter des Vorsitzenden ist,
    \item einem Schriftführer, der zugleich 2. Stellvertreter des Vorsitzenden ist,
    \item zwei Beisitzen.
  \end{enumerate}
  Die nähere Aufgaben- und Zuständigkeitsverteilung kann der AStA nach Amtsantritt in seiner Geschäftsordnung regeln, sonst gilt die allgemeine Geschäftsordnung des STUPA entsprechend.
\end{enumerate}

\subsection{Aufgaben des AStA}
\label{subsec:asta:aufgaben}
\begin{enumerate}[label=(\arabic*)]
  \item Der AStA führt die laufenden Geschäfte der Studierendenschaft.
  \item Der AStA stellt unter Leitung des 1. Stellvertreters einen Finanzplan für ein Haushaltsjahr gemäß den gesetzlichen Vorgaben auf.
  \item Bei unaufschiebbaren Angelegenheiten entscheidet der Vorsitzende anstelle des AStA. Er hat in diesem Fall den AStA unverzüglich zu unterrichten. Der AStA kann die getroffene Entscheidung aufheben, soweit durch ihre Ausführung nicht Rechte Dritter entstanden sind.
  \item Zur Unterstützung des Vorsitzenden bestellt der AStA einen Beauftragten für den Haushalt im Sinne des § 9 LHO, der die Befähigung für den gehobenen Verwaltungsdienst hat oder in vergleichbarer Weise über nachgewiesene Fachkenntnisse im Haushaltsrecht verfügt. Der Haushaltsbeauftragte ist unmittelbar dem Vorsitzenden unterstellt; der Vorsitzende gilt als Leiter der Dienststelle im Sinne des § 9 Abs. 1 S. 2 LHO. Der Finanzreferent arbeitet eng mit dem Beauftragten für den Haushalt zusammen. Erhebt der Haushaltsbeauftragte Widerspruch gegen eine Maßnahme, weil er sie für rechtswidrig oder nach den Grundsätzen der Wirtschaftlichkeit für nicht vertretbar hält, hat der Vorsitzende eine Entscheidung des STUPA herbeizuführen.
\end{enumerate}

\subsection{Wahl, Abwahl und \hl{Ausscheiden} der Mitglieder des AStA}
\label{subsec:asta:mitglieder}
\begin{enumerate}[label=(\arabic*)]
  \item Jedes Mitglied der Studierendenschaft kann sich selbstständig zur Wahl des AStA aufstellen und muss sich bei der ersten Sitzung entsprechender Amtsperiode der Mitglieder des STUPA persönlich vorstellen.
  \item Der Vorsitzende des AStA wird gemäß \cref{subsec:stupa:vorsitzender} gewählt.
  \item Die übrigen Mitglieder des AStA werden nach der Wahl des Vorsitzenden ebenfalls mit einfacher Mehrheit der Mitglieder des STUPA gewählt. \cref{subsec:allg:wahlen} Absatz 1 und 2 gilt entsprechend. Bei mehreren Kandidaten wird jedes Amt einzeln abgestimmt.
  \item \hl{Die Wahl findet in der ersten Sitzung der entsprechenden Amtsperiode des STUPAs statt. Scheidet ein Mitglied des AStA gemäß Absatz 6 aus, so ist der Posten in der nächsten ordentlichen Sitzung des STUPAs gemäß} \cref{subsec:allg:wahlen} \hl{neu zu wählen.}
  \item Mitglieder des AStA können mit Zweidrittelmehrheit der Mitglieder des STUPA abgewählt werden. Ein Mitglied des AStA kann nur abgewählt werden, indem ein neues Mitglied mit Zweidrittelmehrheit der Mitglieder des STUPA gewählt wird. Zu der Sitzung, in der die Abwahl erfolgt, muss mindestens zwei Wochen vor dem Termin eingeladen werden.
  \item \hl{Ein Mitglied des Allgemeinen Studierendenausschusses scheidet aus}
  \begin{enumerate}[label=\alph*)]
    \item \hl{durch Abwahl durch das STUPA entsprechend Absatz 5,}
    \item \hl{durch Exmatrikulation,}
    \item \hl{durch Rücktritt aus wichtigem Grund, der dem Studierendenparlament gegenüber zu erklären ist,}
    \item \hl{durch Tod.}
  \end{enumerate}
\end{enumerate}
