% !TEX root = ../Main.tex
% ==============================================================================
% Fünfter Abschnitt: Vermögensangelegenheiten
% ==============================================================================

\newpage
\chapter{Vermögensangelegenheiten}
\sectionmark{}
\vspace*{-1.5em}
\rule{\textwidth}{0.5mm}
\vspace*{-2em}

\subsection{Grundsätze}
\label{subsec:vermoegen:grundsaetze}
\begin{enumerate}[label=(\arabic*)]
  \item Für die Haushalts- und Wirtschaftsführung sind die für das Land Baden-Württemberg geltenden Vorschriften, insbesondere die §§ 105 bis 111 der Landeshaushaltsordnung, entsprechend anzuwenden. Die Studierendenschaft entscheidet im Rahmen der Rechtsvorschriften unter Berücksichtigung des Grundsatzes der Wirtschaftlichkeit und Sparsamkeit über die zweckmäßige Verwendung der zur Verfügung stehenden Finanzmittel.
  \item Die Studierendenschaft stellt vor Beginn jedes Haushaltsjahres einen Haushaltsplan auf. Er muss alle im Haushaltsjahr zu erwartenden Einnahmen, voraussichtlich zu leistenden Ausgaben und voraussichtlich benötigte Verpflichtungsermächtigungen enthalten und ist in Einnahme und Ausgabe auszugleichen. In den Haushaltsplan dürfen nur die Ausgaben und Verpflichtungsermächtigungen eingestellt werden, die zur Erfüllung der Aufgaben der Studierendenschaft notwendig sind.
  \item Der AStA stellt den Haushaltsplan auf. Der Haushaltsplan ist vom Studierendenparlament zu beschließen. Der Haushaltsplan ist dem Präsidium der Hochschule spätestens einen Monat vor Beginn des Haushaltsjahres zur Genehmigung vorzulegen.
  \item Der AStA stellt unverzüglich nach Ende jedes Haushaltsjahres eine Rechnung auf, die von einer fachkundigen Person mit der Befähigung für den gehobenen Verwaltungsdienst, die nicht mit dem Haushaltsbeauftragten identisch ist, oder der Verwaltung der Hochschule mit ihrem Einvernehmen geprüft wird. Die Beauftragung des Rechnungsprüfers erfolgt durch die Studierendenschaft. Die Entlastung für die Haushalts- und Wirtschaftsführung erteilt das Präsidium der Hochschule.
  \item Für Verbindlichkeiten der Studierendenschaft haftet nur deren Vermögen. Die Haushalts- und Wirtschaftsführung der Studierendenschaft unterliegt der Prüfung durch den Rechnungshof.
  \item Die Studierendenschaft bestreitet die Ausgaben für ihre satzungsgemäßen Aufgaben aus den Beiträgen der Studierenden, aus Zuwendungen Dritter und aus sonstigen Einnahmen. Die Höhe der Beiträge ist für das neue Haushaltsjahr gleichzeitig mit der Feststellung des Haushaltsplanes durch die Beitragssatzung (\cref{subsec:vermoegen:beitraege}) festzusetzen. Sie ist vom Präsidium der Hochschule zu genehmigen, der spätestens einen Monat vor Beginn des Haushaltsjahres über die Festsetzung zu informieren ist.
  \item Der AStA kann im Einvernehmen mit dem Studierendenparlament und im Benehmen mit dem Präsidium der Hochschule festlegen, dass anstelle eines Haushaltsplans ein Wirtschaftsplan geführt wird.
\end{enumerate}

\subsection{Beiträge}
\label{subsec:vermoegen:beitraege}
\begin{enumerate}[label=(\arabic*)]
  \item Die Studierenden leisten angemessene finanzielle Beiträge, die der Studierendenschaft zur Erfüllung ihrer gesetzlichen Aufgaben zur Verfügung stehen (Studierendenschaftsbeitrag).
  \item Das STUPA erlässt eine Beitragssatzung. Sie muss insbesondere Bestimmungen über die Beitragspflicht, die Höhe des Beitrags und die Beitragsfälligkeit enthalten. Bei der Festsetzung der Beitragshöhe sind die sozialen Belange der Studierenden zu berücksichtigen.
\end{enumerate}

\subsection{Wirtschaftliche Betätigung}
\label{subsec:vermoegen:wirtschaftliche-betätigung}
\begin{enumerate}[label=(\arabic*)]
  \item Eine wirtschaftliche Betätigung der Studierendenschaft ist nur innerhalb der ihr obliegenden Aufgaben und nur insoweit zulässig, als die Betätigung nach Art und Umfang in einem angemessenen Verhältnis zur Leistungsfähigkeit der Studierendenschaft und zum voraussichtlichen Bedarf steht.
  \item Im Falle der Gründung eines oder Beteiligung an einem Unternehmen in Privatrechtsform muss darüber hinaus der von der Studierendenschaft angestrebte Zweck nicht besser und wirtschaftlicher auf andere Weise zu erreichen sein, die Einzahlungsverpflichtung der Studierendenschaft muss auf einen bestimmten Betrag begrenzt sein, die Studierendenschaft muss einen angemessenen Einfluss, insbesondere im Aufsichtsrat oder in einem entsprechenden Überwachungsorgan erhalten und es muss gewährleistet sein, dass der Jahresabschluss und der Lagebericht, soweit nicht weitergehende gesetzliche Vorschriften gelten oder andere gesetzliche Vorschriften entgegenstehen, in entsprechender Anwendung der Vorschriften des Dritten Buchs des Handelsgesetzbuchs für große Kapitalgesellschaften aufgestellt und geprüft wird.
  \item Die Beteiligung der Studierendenschaft an wirtschaftlichen Unternehmen oder die Gründung wirtschaftlicher Unternehmen bedarf der vorherigen Zustimmung des Präsidiums der Hochschule.
  \item Darlehen darf die Studierendenschaft nicht aufnehmen oder vergeben; sie darf ein Girokonto auf Guthabenbasis führen.
  \item Beim Abschluss von Werkverträgen und bei sonstigen Beschaffungsvorgängen sind die geltenden Vergabevorschriften zu berücksichtigen.
\end{enumerate}

\subsection{Haushaltsplan und Finanzordnung}
\label{subsec:vermoegen:haushaltsplan-finanzordnung}
Die Studierendenschaft regelt das Nähere über die Aufstellung und Ausführung des Haushaltsplanes, die Haushalts-, Wirtschafts- und Kassenführung sowie die Rechnungslegung in der Finanzordnung.

\subsection{Arbeitsentgelte und Aufwandsentschädigungen}
\label{subsec:vermoegen:arbeitsentgelte-aufwandsentschaedigungen}
\begin{enumerate}[label=(\arabic*)]
  \item Beschäftigte der Studierendenschaft unterliegen derselben Tarifbindung wie Beschäftigte der Hochschule. Die unbefristete Einstellung von Personal ist nur zulässig, wenn dafür im Haushaltsplan der Studierendenschaft ausdrücklich Mittel bereitgestellt wurden und diese Mittel ausreichend sind, alle durch das Personal entstehenden Kosten zu decken. Stellen sind öffentlich auszuschreiben. Für die Personalauswahl gilt der Grundsatz der Bestenauslese.
  \item Die Mitglieder in den Organen der Studierendenschaft üben ihre Tätigkeit ehrenamtlich aus. Das Studierendenparlament kann für die Mitglieder des Allgemeinen Studierendenausschusses, \hl{für die Mitglieder von Referatsleitungen} sowie für die Mitglieder des Studierendenparlamentes eine angemessene Aufwandsentschädigung festsetzen. \hl{Näheres regelt die Geschäftsordnung (GO) des Studierendenparlamentes.}
  \label{subsec:vermoegen:arbeitsentgelte-aufwandsentschaedigungen:abs:2}
\end{enumerate}
